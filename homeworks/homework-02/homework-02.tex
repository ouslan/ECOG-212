\documentclass{article}

\usepackage{fancyhdr}
\usepackage{extramarks}
\usepackage{amsmath}
\usepackage{amsthm}
\usepackage{amsfonts}
\usepackage{tikz}
\usepackage[plain]{algorithm}
\usepackage{algpseudocode}

\usetikzlibrary{automata,positioning}

%
% Basic Document Settings
%

\topmargin=-0.45in
\evensidemargin=0in
\oddsidemargin=0in
\textwidth=6.5in
\textheight=9.0in
\headsep=0.25in

\linespread{1.1}

\pagestyle{fancy}
\lhead{\hmwkAuthorName}
\chead{\hmwkClass\ (\hmwkClassInstructor): \hmwkTitle}
\rhead{\firstxmark}
\lfoot{\lastxmark}
\cfoot{\thepage}

\renewcommand\headrulewidth{0.4pt}
\renewcommand\footrulewidth{0.4pt}

\setlength\parindent{0pt}

%
% Create Problem Sections
%

\newcommand{\enterProblemHeader}[1]{
	\nobreak\extramarks{}{Problem \arabic{#1} continued on next page\ldots}\nobreak{}
	\nobreak\extramarks{Problem \arabic{#1} (continued)}{Problem \arabic{#1} continued on next page\ldots}\nobreak{}
}

\newcommand{\exitProblemHeader}[1]{
	\nobreak\extramarks{Problem \arabic{#1} (continued)}{Problem \arabic{#1} continued on next page\ldots}\nobreak{}
	\stepcounter{#1}
	\nobreak\extramarks{Problem \arabic{#1}}{}\nobreak{}
}

\setcounter{secnumdepth}{0}
\newcounter{partCounter}
\newcounter{homeworkProblemCounter}
\setcounter{homeworkProblemCounter}{1}
\nobreak\extramarks{Problem \arabic{homeworkProblemCounter}}{}\nobreak{}

%
% Homework Problem Environment
%
% This environment takes an optional argument. When given, it will adjust the
% problem counter. This is useful for when the problems given for your
% assignment aren't sequential. See the last 3 problems of this template for an
% example.
%
\newenvironment{homeworkProblem}[1][-1]{
	\ifnum#1>0
		\setcounter{homeworkProblemCounter}{#1}
	\fi
	\section{Problem \arabic{homeworkProblemCounter}}
	\setcounter{partCounter}{1}
	\enterProblemHeader{homeworkProblemCounter}
}{
	\exitProblemHeader{homeworkProblemCounter}
}

%
% Homework Details
%   - Title
%   - Due date
%   - Class
%   - Section/Time
%   - Instructor
%   - Author
%

\newcommand{\hmwkTitle}{Problem Set\ \#1}
\newcommand{\hmwkDueDate}{Jun 15, 2025}
\newcommand{\hmwkClass}{ECON 124}
\newcommand{\hmwkClassInstructor}{Dr. Deniz Baglan}
\newcommand{\hmwkAuthorName}{\textbf{Alejandro Ouslan}}

%
% Title Page
%

\title{
	\vspace{2in}
	\textmd{\textbf{\hmwkClass:\ \hmwkTitle}}\\
	\normalsize\vspace{0.1in}\small{Due\ on\ \hmwkDueDate}\\
	\vspace{0.1in}\large{\textit{\hmwkClassInstructor}}
	\vspace{3in}
}

\author{\hmwkAuthorName}
\date{}

\renewcommand{\part}[1]{\textbf{\large Part \Alph{partCounter}}\stepcounter{partCounter}\\}

%
% Various Helper Commands
%

% Useful for algorithms
\newcommand{\alg}[1]{\textsc{\bfseries \footnotesize #1}}

% For derivatives
\newcommand{\deriv}[1]{\frac{\mathrm{d}}{\mathrm{d}x} (#1)}

% For partial derivatives
\newcommand{\pderiv}[2]{\frac{\partial}{\partial #1} (#2)}

% Integral dx
\newcommand{\dx}{\mathrm{d}x}

% Alias for the Solution section header
\newcommand{\solution}{\textbf{\large Solution}}

% Probability commands: Expectation, Variance, Covariance, Bias
\newcommand{\E}{\mathrm{E}}
\newcommand{\Var}{\mathrm{Var}}
\newcommand{\Cov}{\mathrm{Cov}}
\newcommand{\Bias}{\mathrm{Bias}}

\begin{document}

\maketitle

\pagebreak

% Homework problem 1
\begin{homeworkProblem}
  Consider two least-square Regressions
  $$y = X_1 \beta_1 + \epsilon$$
  and 
  $$y = X_1 \beta_1 + X_2 \beta_2 \epsilon$$
  Let $R_1^2$ and $R_2^2$ be the R+squared from the two regressions, respectively. Show
  that $R_2^2 \ge R_1^2$
\end{homeworkProblem}

% Homework Problem 2 
\begin{homeworkProblem}
  Use the cps09mar data for this question. The data set and the description file is 
  attached under the assignment of Canvas. Estimate a log wage regression for the subsample 
  of white male Hispanics. In addition to education, experience, and its square, include a set of binary 
  variables for Northeast, South and West so that Midwest is the exuded group. For marital status, create 
  variables for married, widow or divorced, and separated, so that single (never married) is the excluded 
  group.
\end{homeworkProblem}

% Homework Problem 3 
\begin{homeworkProblem}
  The data koop tobias subsample is extracted from koop and tobias (2004) study
  of the relationship between wages and education, ability, and family characterises. 
  Et $X_1$ equal a constant, education, experience, and ability. Let $X_2$ contain
  the mothers education, the fathers education, and the number of siblings. Let $y$ 
  e the log of the hourly wage. Show you regression output

  \begin{enumerate}
    \item Compute the least square regression coefficients in the regression of $y$ on $X_1$.
      Report the coefficients. 
    \item Compute the least square regression coefficients in the regression of $y$ on $X_1$ 
      and $X_2$. Report the coefficients.
    \item Regress each of the three variables in $X_2$ on all the variables in  $X_1$ and copete 
      the residuals form each regression. Arrange these new variables in the (15X3) matrix $X_2^\star$. 
      What are the sample means of these three variables?
    \item Compute the $R^2$ for the regress8ion of $y$ on $X_1$ and $X_2$. Repeat the computation for the 
      case in which the constant term is omitted from $X_1$. What happens to $R^2$? 
    \item Compute the adjusted $R^2$ for the full regression including the constant term. 
  \end{enumerate}
\end{homeworkProblem}

% Homework Problem 4
Suppose that you have two independent unbiased estimators of the same parameter $\theta$, say 
$\hat{\theta}_1$ and $\hat{\theta}_2$, with different variances $v_1$ and $v_2$, What linear combination 
$\hat{\theta} = c_1\hat{\theta_1} + c_2\hat{\theta}_2$ is the minimum variance unbiased estimator of 
$\theta$?

% Homework Problem 5 
Analyze the properties of the Least Squares (LS) estimator of the slope coefficient in a simple 
linear regression model using the Monte Carl method. Specifically, the true model is given by 
$$y_i = 1 + 2x_i + \epsilon_i$$

Where the intercept is 1, and the slope coefficient is 2. The errors $\epsilon_i$ are normally 
distributed with mean zero and variance 3, $\epsilon \sim N(0,6)$. The independent variable $x$
is uniformly distributed between 0 and 6, $X \sim (0,6)$.

Conduct the following experiment: Draw 1000 samples of sizes 25,50, and 100 from the 
distribution of the error term and the predictor variable. For each sample, estimate
the slope coefficient using the LS estimator. Then, analyze the 

\begin{enumerate}
  \item Unbiasedness, variance, and the shape of the distribution of the LS estimator of the slope 
    coefficient by plotting the distribution for each sample size. 
  \item Compare the variance of the estimated slope coefficient obtained form the Monte Carlo simulation with 
    the variance obtained using the asymptotic formula and the variance obtained using a bootstrap method 
\end{enumerate}

\end{document}
