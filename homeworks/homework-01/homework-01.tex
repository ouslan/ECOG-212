\documentclass{article}

\usepackage{fancyhdr}
\usepackage{extramarks}
\usepackage{amsmath}
\usepackage{amsthm}
\usepackage{amsfonts}
\usepackage{tikz}
\usepackage[plain]{algorithm}
\usepackage{algpseudocode}

\usetikzlibrary{automata,positioning}

%
% Basic Document Settings
%

\topmargin=-0.45in
\evensidemargin=0in
\oddsidemargin=0in
\textwidth=6.5in
\textheight=9.0in
\headsep=0.25in

\linespread{1.1}

\pagestyle{fancy}
\lhead{\hmwkAuthorName}
\chead{\hmwkClass\ (\hmwkClassInstructor): \hmwkTitle}
\rhead{\firstxmark}
\lfoot{\lastxmark}
\cfoot{\thepage}

\renewcommand\headrulewidth{0.4pt}
\renewcommand\footrulewidth{0.4pt}

\setlength\parindent{0pt}

%
% Create Problem Sections
%

\newcommand{\enterProblemHeader}[1]{
	\nobreak\extramarks{}{Problem \arabic{#1} continued on next page\ldots}\nobreak{}
	\nobreak\extramarks{Problem \arabic{#1} (continued)}{Problem \arabic{#1} continued on next page\ldots}\nobreak{}
}

\newcommand{\exitProblemHeader}[1]{
	\nobreak\extramarks{Problem \arabic{#1} (continued)}{Problem \arabic{#1} continued on next page\ldots}\nobreak{}
	\stepcounter{#1}
	\nobreak\extramarks{Problem \arabic{#1}}{}\nobreak{}
}

\setcounter{secnumdepth}{0}
\newcounter{partCounter}
\newcounter{homeworkProblemCounter}
\setcounter{homeworkProblemCounter}{1}
\nobreak\extramarks{Problem \arabic{homeworkProblemCounter}}{}\nobreak{}

%
% Homework Problem Environment
%
% This environment takes an optional argument. When given, it will adjust the
% problem counter. This is useful for when the problems given for your
% assignment aren't sequential. See the last 3 problems of this template for an
% example.
%
\newenvironment{homeworkProblem}[1][-1]{
	\ifnum#1>0
		\setcounter{homeworkProblemCounter}{#1}
	\fi
	\section{Problem \arabic{homeworkProblemCounter}}
	\setcounter{partCounter}{1}
	\enterProblemHeader{homeworkProblemCounter}
}{
	\exitProblemHeader{homeworkProblemCounter}
}

%
% Homework Details
%   - Title
%   - Due date
%   - Class
%   - Section/Time
%   - Instructor
%   - Author
%

\newcommand{\hmwkTitle}{Problem Set\ \#1}
\newcommand{\hmwkDueDate}{Jun 5, 2025}
\newcommand{\hmwkClass}{ECON 124}
\newcommand{\hmwkClassInstructor}{Dr. Deniz Baglan}
\newcommand{\hmwkAuthorName}{\textbf{Alejandro Ouslan}}

%
% Title Page
%

\title{
	\vspace{2in}
	\textmd{\textbf{\hmwkClass:\ \hmwkTitle}}\\
	\normalsize\vspace{0.1in}\small{Due\ on\ \hmwkDueDate}\\
	\vspace{0.1in}\large{\textit{\hmwkClassInstructor}}
	\vspace{3in}
}

\author{\hmwkAuthorName}
\date{}

\renewcommand{\part}[1]{\textbf{\large Part \Alph{partCounter}}\stepcounter{partCounter}\\}

%
% Various Helper Commands
%

% Useful for algorithms
\newcommand{\alg}[1]{\textsc{\bfseries \footnotesize #1}}

% For derivatives
\newcommand{\deriv}[1]{\frac{\mathrm{d}}{\mathrm{d}x} (#1)}

% For partial derivatives
\newcommand{\pderiv}[2]{\frac{\partial}{\partial #1} (#2)}

% Integral dx
\newcommand{\dx}{\mathrm{d}x}

% Alias for the Solution section header
\newcommand{\solution}{\textbf{\large Solution}}

% Probability commands: Expectation, Variance, Covariance, Bias
\newcommand{\E}{\mathrm{E}}
\newcommand{\Var}{\mathrm{Var}}
\newcommand{\Cov}{\mathrm{Cov}}
\newcommand{\Bias}{\mathrm{Bias}}

\begin{document}

\maketitle

\pagebreak

% Homework problem 1
\begin{homeworkProblem}
  Let 
    $A = \begin{bmatrix} 1 & 2 & 3 \\ 4 & 5 & 6 \end{bmatrix}$, 
    $B = \begin{bmatrix} 1 & -1 \\ 0 & 1\end{bmatrix}$, 
    $C = \begin{bmatrix} -1 & 0 \\ 1 & 1 \\ 0 & 1\end{bmatrix}$,
    $D = \begin{bmatrix} -3 & -2 & -1 \\ 1 & 2 & 3\end{bmatrix}$.
    Find the following:
    \begin{enumerate}
      \item $A - C'$
      \item $C' + 3D$
      \item $CB$
      \item $D'D$
    \end{enumerate}
\end{homeworkProblem}

% Problem 2 
\begin{homeworkProblem}
  Given the square matrices:
  $$
  A = \begin{bmatrix}
    3 & -1 & 2 \\ 1 & 0 & 3 \\ 3 & -2 & -5
    \end{bmatrix}, B = \begin{bmatrix}
    3 & -6 & -3 \\ 7 & -14 & -7 \\ -1 & 2 & 1
  \end{bmatrix}
    $$
    Verify that $AB=0$
\end{homeworkProblem}

% Problem 3 
\begin{homeworkProblem}
  The following table shows the number of personnel, in thousands, 
  in three branches of the U.S. Army in 2001, and the changes in 2002 and 2003.

\begin{table}[h!]
  \centering
  \begin{tabular}{|c|c|c|c|}
    \hline
     & Active Duty & Reserve & National Guard \\
    \hline
    2001 & 75 & 35 & 60 \\
    \hline
    Change in 2002 & 5 & -15 & 2 \\
    \hline
    2003 & -12 & 5 & -17 \\
    \hline
  \end{tabular}
\end{table}
Use matrix algebra to find the number of personel in each branch 
\begin{enumerate}
  \item peronnel 2001
  \item Personnel 2002
\end{enumerate}

\end{homeworkProblem}

% Problem 4
\begin{homeworkProblem}
  \begin{enumerate}
    \item Suppose $b_1$ is the least squares estimator of the slope coefficient 
      in a regression of $Y$ on $X$ and $b_2$ is the slope coefficient estimator in a "reverse"
      regression of $X$ on $Y$. Show that $R^2 = b_1 b_2$ where $R$ is the correlation between 
      $Y$ and $X$. 
    \item From a sample of 200 observation the following quantities were calculated:
      $$
      \sum X =11, \sum Y = 20, \sum X^2 = 12, \sum XY= 22, \sum Y^2 =84
      $$
      Estimate both regression equations an calculate $R^2$. Calculate the standard error 
      of $b_1$.
  \end{enumerate}
\end{homeworkProblem}

% Problem 5
\begin{homeworkProblem}
  Consider the equation $Y = \beta_0 + \beta_1 X_1 + \beta_2 X_2$, where 
  $X_1 \sim (\mu_{X_1}, \sigma_{X_1}^2)$ is independent of $X_2 \sim (\mu_{X_2}, \sigma_{X_2}^2)$ 
  with this information answer the following questions:
  \begin{enumerate}
    \item What is the expected value of $X_1$? What is the expected value of  $X_2$?
    \item What is the variance of $X_1?$ what is the variance of  $X_2$.
    \item What is the expected value of $Y$?
    \item What is the variance of $Y$?
    \item What is the marginal effect of $X_1$ on $Y$? What is the marginal effect of 
      $X_2$ on $Y$?
  \end{enumerate}
\end{homeworkProblem}

% Problem 6
\begin{homeworkProblem}
  Consider an independent random sample of data of size  $n$ drawn from the continuous distribution 
  of $X \sim N(\mu,\sigma^2)$. Suppose for whatever reason, we do not like the fist observtion and we 
  propose the following estimator of $\mu$:
  $$
  \bar{X} = \frac{x_2 + x_3 + \ldots + x_n}{n}
  $$
  \begin{enumerate}
    \item what is expected value of $\bar{x}$
    \item What is the bias of $\bar{x}$?
    \item What is the variance of $\bar{x}$?
    \item What is the mean square error of $\bar{x}$?
    \item What happens to the results in parts (a-d) when the sample size $n$ tends to infinity? what can be said 
      about this estimator in this scenario?
  \end{enumerate}
\end{homeworkProblem}

% Problem 7
\begin{homeworkProblem}
  Consider the following multiple linear regression model
  $$y = X\beta + u$$
  where $y$ is $n \times 1$, $X$ is $n \times k$ and u is $n \times 1$ 
  such that $u|x \sim N(0, \sigma^2 I_n)$. Write $Y= \hat{Y} + \hat{u}$,where 
  $\hat{y} = X\hat{\beta}$ is the least squares predicted values. 
  \begin{enumerate}
    \item Show that $(\hat{\beta} - \beta)= Au$ and $\hat{u} = Mu$, what are your $A$ and $M$?
    \item Show that $\bar{y} = $ the mean of the predicted values $\hat{y}$
    \item Show that $X'\hat{u} = 0,\hat{y}\hat{u}=0$
    \item Derive $R^2$ for the model where the first column of $X$ has a constant.
  \end{enumerate}
\end{homeworkProblem}

\end{document}
