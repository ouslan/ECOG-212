\documentclass{article}

\usepackage{fancyhdr}
\usepackage{extramarks}
\usepackage{amsmath}
\usepackage{amsthm}
\usepackage{amsfonts}
\usepackage{tikz}
\usepackage[plain]{algorithm}
\usepackage{algpseudocode}

\usetikzlibrary{automata,positioning}

%
% Basic Document Settings
%

\topmargin=-0.45in
\evensidemargin=0in
\oddsidemargin=0in
\textwidth=6.5in
\textheight=9.0in
\headsep=0.25in

\linespread{1.1}

\pagestyle{fancy}
\lhead{\hmwkAuthorName}
\chead{\hmwkClass\ (\hmwkClassInstructor): \hmwkTitle}
\rhead{\firstxmark}
\lfoot{\lastxmark}
\cfoot{\thepage}

\renewcommand\headrulewidth{0.4pt}
\renewcommand\footrulewidth{0.4pt}

\setlength\parindent{0pt}

%
% Create Problem Sections
%

\newcommand{\enterProblemHeader}[1]{
	\nobreak\extramarks{}{Problem \arabic{#1} continued on next page\ldots}\nobreak{}
	\nobreak\extramarks{Problem \arabic{#1} (continued)}{Problem \arabic{#1} continued on next page\ldots}\nobreak{}
}

\newcommand{\exitProblemHeader}[1]{
	\nobreak\extramarks{Problem \arabic{#1} (continued)}{Problem \arabic{#1} continued on next page\ldots}\nobreak{}
	\stepcounter{#1}
	\nobreak\extramarks{Problem \arabic{#1}}{}\nobreak{}
}

\setcounter{secnumdepth}{0}
\newcounter{partCounter}
\newcounter{homeworkProblemCounter}
\setcounter{homeworkProblemCounter}{1}
\nobreak\extramarks{Problem \arabic{homeworkProblemCounter}}{}\nobreak{}

%
% Homework Problem Environment
%
% This environment takes an optional argument. When given, it will adjust the
% problem counter. This is useful for when the problems given for your
% assignment aren't sequential. See the last 3 problems of this template for an
% example.
%
\newenvironment{homeworkProblem}[1][-1]{
	\ifnum#1>0
		\setcounter{homeworkProblemCounter}{#1}
	\fi
	\section{Problem \arabic{homeworkProblemCounter}}
	\setcounter{partCounter}{1}
	\enterProblemHeader{homeworkProblemCounter}
}{
	\exitProblemHeader{homeworkProblemCounter}
}

%
% Homework Details
%   - Title
%   - Due date
%   - Class
%   - Section/Time
%   - Instructor
%   - Author
%

\newcommand{\hmwkTitle}{Asignacion\ \#1}
\newcommand{\hmwkDueDate}{Septiembre 5, 2024}
\newcommand{\hmwkClass}{ESMA 6000}
\newcommand{\hmwkClassInstructor}{Israel Almodovar}
\newcommand{\hmwkAuthorName}{\textbf{Alejandro Ouslan}}

%
% Title Page
%

\title{
	\vspace{2in}
	\textmd{\textbf{\hmwkClass:\ \hmwkTitle}}\\
	\normalsize\vspace{0.1in}\small{Due\ on\ \hmwkDueDate}\\
	\vspace{0.1in}\large{\textit{\hmwkClassInstructor}}
	\vspace{3in}
}

\author{\hmwkAuthorName}
\date{}

\renewcommand{\part}[1]{\textbf{\large Part \Alph{partCounter}}\stepcounter{partCounter}\\}

%
% Various Helper Commands
%

% Useful for algorithms
\newcommand{\alg}[1]{\textsc{\bfseries \footnotesize #1}}

% For derivatives
\newcommand{\deriv}[1]{\frac{\mathrm{d}}{\mathrm{d}x} (#1)}

% For partial derivatives
\newcommand{\pderiv}[2]{\frac{\partial}{\partial #1} (#2)}

% Integral dx
\newcommand{\dx}{\mathrm{d}x}

% Alias for the Solution section header
\newcommand{\solution}{\textbf{\large Solution}}

% Probability commands: Expectation, Variance, Covariance, Bias
\newcommand{\E}{\mathrm{E}}
\newcommand{\Var}{\mathrm{Var}}
\newcommand{\Cov}{\mathrm{Cov}}
\newcommand{\Bias}{\mathrm{Bias}}

\begin{document}

\maketitle

\pagebreak

% Homework problem 1
\begin{homeworkProblem}
	Using your own words define the following concepts:
	\begin{itemize}
		\item \textbf{Sample Space:}
		      The sample space is the set of all possible outcomes of an experiment. It is to say everythig that could happen
		      in the experiment
		\item \textbf{Countable Sets (give an example):}
		      A countable set is a set that can be put in a one-to-one correspondence with the set of natural numbers $\mathbb{N}$.
		      An example of a countable set is the set would be the set of possible lottery numbers given that we can express or quantify
		      the set of numbers that can be drawn.
		\item \textbf{$\sigma$-algebra $\mathcal{B}$:}
		      $\sigma$-algebra is a collection of combinations of sets or events of the sample space.
		\item \textbf{The following triplet $(\mathcal{S}, \mathcal{B}, P)$:}
		      The triplet $(\mathcal{S}, \mathcal{B}, P)$ is given the sample space $\mathcal{S}$, the event/set of events $\sigma$-algebra
		      $\mathcal{B}$ has the probability measure $P$.
	\end{itemize}
\end{homeworkProblem}

% Homework problem 2
\begin{homeworkProblem}

	For each of the following experiments, describe the sample space.

	\begin{itemize}
		\item Toss a coin four times.\\
		      The sample space is the set of all possible outcomes of the experiment. An example of the events that are contain in the sample space
		      are $\{HHHH, HHHT, \dots, TTTT\}$.
		\item Count the number of insects-damaged leaves on a plant. \\
		      The sample space is the set of all possible counts of insects-damaged leaves on a plant. An example asumming that there are 10 plants
		      the sample space would be $\{0, 1, 2, \dots, 10\}$.
		\item Measure the lifetime (in hours) of a particular brand of lights bulbs.
		      The sample space is the set of all possible lifetimes (in hours) of a particular brand of lights. An example given the brand \textit{generic-x}
		      and we looked at 5 bulbs the sample space would be $\{0, \dots, 1000, \dots, \infty\}$.
		\item Record the weights of 10-day-old babies. \\
		      The sample space is the set of all possible weights of 10-day-old babies. An example given the weights of 10-day-old babies in grams the sample
		      space would be $\{0, \dots, 1000, \dots, \infty\}$.
		\item Observe the proportion of defective in a shipment of electreonic components. \\
		      The sample space is the set of all possible proportions of defective in a shipment of electronic components. An example given the proportion of defective
		      in a shipment of electronic components the sample space would be $\{0, \dots, 1\}$.

	\end{itemize}
\end{homeworkProblem}

% Homework problem 3
\begin{homeworkProblem}
	Supose that $A \subset B$. Show that $B^c \subset A^c$
	\begin{proof}
		\begin{align*}
		\end{align*}
	\end{proof}
\end{homeworkProblem}

% Homework problem 4
\begin{homeworkProblem}
	Let $S$ be a sample space. Show that the collection $\mathcal{B} = \{\emptyset, S\}$ is a $\sigma$-algebra.
	\begin{proof}
		\begin{align*}
		\end{align*}
	\end{proof}
\end{homeworkProblem}

% Homework problem 5
\begin{homeworkProblem}
	Let $\Omega = \{1, 2, 3\}$ be the sample space. Let $\mathcal{B} = \{\{1\}, \{2, 3\}, \emptyset, \Omega\}$.
	be a collection of $S$.
	\begin{itemize}
		\item Verify that $\mathcal{B_1}$ and $\mathcal{B_2}$ are $\sigma$-algebras.
		\item Verify that $\mathcal{B_1} \cap \mathcal{B_2}$ is a $\sigma$-algebra.
		\item Verify that $\mathcal{B_1} \cup \mathcal{B_2}$ is not a $\sigma$-algebra.
		\item Discuss your results from (b) and (c).
	\end{itemize}
\end{homeworkProblem}

% Homework problem 6
\begin{homeworkProblem}
	One ball is to be selected from a box containing red, white, blue, yellow, and green
	balls. If the probability that the selected ball will be red is $\frac{1}{5}$, and the probability that it will be
	white is $\frac{2}{5}$, what is the probability that it will be blue, yellow, or green?
\end{homeworkProblem}

% Homework problem 7
\begin{homeworkProblem}
	If $P(A) = \frac{1}{3}$  and $P(B^c) = \frac{1}{2}$, can $A$ and $B$ be disjoint? Explain.
\end{homeworkProblem}

% Homework problem 8
\begin{homeworkProblem}
	Prove that every two events $A$ and $B$, the probability that exactly one of the two events will
	occur is given by the expression
	$$P(A) + P(B) - 2P(A \cap B)$$
\end{homeworkProblem}

% Homework problem 9
\begin{homeworkProblem}
	For events $A$ and $B$, find formulas for the probabilities of the following events in terms of the quantities
	$P(A)$, $P(B)$, and $P(A \cap B)$:
	\begin{itemize}
		\item either $A$ or $B$ or both occur.
		\item either $A$ or $B$ but not both occur.
		\item at least one of $A$ or $B$.
		\item at most one of $A$ or $B$.
	\end{itemize}
\end{homeworkProblem}

\end{document}


