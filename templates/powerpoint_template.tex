\documentclass[12pt]{beamer}

% ****************
% ***** INFO *****
% ****************
\usepackage[english]{babel}
\title[]{Assignment 1}
\subtitle{ECAG-6665: Applied Econometrics}
\author[Name Surname]{Alejandro Ouslan}
\institute[UPR.png]{University of Puerto Rico}
\date{} % or \today

% *******************
% ***** PROJECT *****
% *******************
% main color: to black
\definecolor{main}{HTML}{000000}
\setbeamercolor{structure}{fg=main}

% *****************
% ***** THEME *****
% *****************
\usepackage{helvet}
\renewcommand{\familydefault}{\sfdefault}
\setbeamertemplate{frametitle continuation}{\gdef\beamer@frametitle{}}
\setbeamertemplate{footline}{}
\setbeamertemplate{navigation symbols}{}
\usepackage{csquotes}
\usepackage[backend=biber,style=numeric]{biblatex}
\addbibresource{sample.bib} % Link to the bibliography file

% *****************
% ***** CODE *****
% *****************
\usepackage{listings}
\lstdefinestyle{java}{
	backgroundcolor=\color{white},
	basicstyle=\ttfamily\scriptsize,
	breaklines=true,
	commentstyle=\color{gray},
	keywordstyle=\color{blue},
	stringstyle=\color{magenta},
	identifierstyle=\color{black},
	numberstyle=\color{gray},
	language=Java
}
\lstdefinestyle{cpp}{
	backgroundcolor=\color{white},
	basicstyle=\ttfamily\scriptsize,
	breaklines=true,
	commentstyle=\color{gray},
	keywordstyle=\color{blue},
	stringstyle=\color{magenta},
	identifierstyle=\color{black},
	numberstyle=\color{gray},
	language=C++
}
\lstdefinestyle{py}{
	backgroundcolor=\color{white},
	basicstyle=\ttfamily\scriptsize,
	breaklines=true,
	commentstyle=\color{gray},
	keywordstyle=\color{blue},
	stringstyle=\color{magenta},
	language=Python
}
\lstdefinestyle{js}{
	backgroundcolor=\color{white},
	basicstyle=\ttfamily\scriptsize,
	breaklines=true,
	commentstyle=\color{gray},
	keywordstyle=\color{blue},
	stringstyle=\color{magenta},
	identifierstyle=\color{black},
	numberstyle=\color{gray},
	language=JavaScript,
	escapechar=@
}
\lstdefinestyle{sh}{
	basicstyle=\ttfamily\scriptsize,
	breaklines=true,
	commentstyle=\color{gray},
	keywordstyle=\color{blue},
	stringstyle=\color{magenta},
	identifierstyle=\color{black},
	numberstyle=\color{gray},
	language=bash
}

% **********************
% ***** ALGORITHMS *****
% **********************
\usepackage{algorithm}
\usepackage{algpseudocode}

% *****************
% ***** UTILS *****
% *****************
\usepackage{xcolor}

% ********************
% ***** DOCUMENT *****
% ********************
\begin{document}

% **********************
% ***** TITLEPAGE ******
% **********************
\begin{frame}{}
	\vspace{\fill}

	\includegraphics[width=0.16\linewidth]{UPR.png}

	\vspace{\fill}

	\Large
	\color{main}
	\inserttitle

	\medskip

	\large
	\color{black}
	\insertsubtitle

	\vspace{\fill}

	\footnotesize
	\insertinstitute

	\vspace{\fill}

	\textbf{Author:} \insertauthor

	\medskip

	\insertdate

	\vspace{\fill}
\end{frame}

% *****************
% ***** START *****
% *****************
\begin{frame}[allowframebreaks]{Typography}
	\begin{itemize}
		\item \textbf{Bold}
		\item \textit{Italic}
		\item \texttt{Monospaced}
		\item \underline{Underlined}
		\item \href{https://example.com/}{\underline{\color{main}{Link}}}
	\end{itemize}
\end{frame}

\begin{frame}[allowframebreaks]{Lists}
	Itemize:

	\begin{itemize}
		\item \textbf{Item 1:} Example of an item in an itemize list.
		\item \textbf{Item 2:} Another item demonstrating the use of itemize.
		\item \textbf{Item 3:} Yet another example to show list formatting.
	\end{itemize}
\end{frame}

\begin{frame}[allowframebreaks]{Lists}
	Enumerate:

	\begin{enumerate}
		\item \textbf{Item 1:} Example of an enumerated item.
		\item \textbf{Item 2:} Another enumerated item to illustrate the format.
		\item \textbf{Item 3:} Final item in the enumerated list.
	\end{enumerate}
\end{frame}

\begin{frame}[allowframebreaks]{Blocks}
	\begin{block}{Note}
		This is a note block that provides additional information or comments about the topic discussed.
	\end{block}

	\begin{exampleblock}{Example}
		Here is an example block that demonstrates how to use this particular format for presenting examples.
	\end{exampleblock}

	\begin{alertblock}{Alert}
		This alert block highlights important warnings or critical information that needs attention.
	\end{alertblock}
\end{frame}

\begin{frame}[allowframebreaks]{Mathematics}
	Let $f$ and $g$ be functions where $g: A \to B$ and $f: B \to \mathbb{R}$. Suppose $g$ is differentiable at $x \in A$ and $f$ is differentiable at $g(x)$. Then the composition function $f \circ g$ is differentiable at $x$, and its derivative is given by:

	\begin{equation}
		\frac{d}{dx} \left[ f(g(x)) \right] = f'(g(x)) \cdot g'(x).
		\label{eq:1}
	\end{equation}
\end{frame}

\begin{frame}[allowframebreaks, fragile]{Code}
	\lstset{style=py}
	\begin{lstlisting}
# Model definition
model = Sequential()
model.add(Flatten(input_shape=(28, 28)))
model.add(Dense(128, activation='relu'))
model.add(Dense(64, activation='relu'))
model.add(Dense(10, activation='softmax'))
model.compile(optimizer='adam', loss='categorical_crossentropy', metrics=['accuracy'])

# Model Training
history = model.fit(X_train, y_train, epochs=10, batch_size=32, validation_split=0.2)

# Model Evaluation
test_loss, test_acc = model.evaluate(X_test, y_test)
\end{lstlisting}
\end{frame}

\begin{frame}[allowframebreaks]{Algorithms}
	\begin{algorithmic}
		\Procedure{BubbleSort}{$array$}
		\State $n \gets \textbf{length}(array)$
		\For{$i \gets 0$ \textbf{to} $n - 1$}
		\For{$j \gets 0$ \textbf{to} $n - i - 2$}
		\If{$array[j] > array[j + 1]$}
		\State \textbf{swap}($array[j], array[j + 1]$)
		\EndIf
		\EndFor
		\EndFor
		\EndProcedure
		\label{alg:1}
	\end{algorithmic}
\end{frame}

\begin{frame}[allowframebreaks]{Images}
	\begin{figure}
		\centering
		\includegraphics[width=0.5\linewidth]{example-image}
		\caption{Example of single image.}
		\label{fig:1}
	\end{figure}
\end{frame}

\begin{frame}[allowframebreaks]{Images}
	\begin{table}
		\centering
		\begin{tabular}{cc}
			\includegraphics[width=0.4\linewidth]{UPR.png} &
			\includegraphics[width=0.4\linewidth]{example-image} \\
		\end{tabular}
		\caption{Example of multiple images.}
		\label{tab:1}
	\end{table}
\end{frame}

\begin{frame}[allowframebreaks]{Tables}
	\begin{table}
		\centering
		\begin{tabular}{|c|c|c|c|}
			\hline
			\textbf{ID} & \textbf{Name} & \textbf{Surname} & \textbf{Birthdate} \\
			\hline
			1           & John          & Doe              & 1990-05-15         \\
			\hline
			2           & Jane          & Smith            & 1985-11-30         \\
			\hline
			3           & Alice         & Johnson          & 2000-07-22         \\
			\hline
			4           & Bob           & Brown            & 1992-02-09         \\
			\hline
			5           & Carol         & Davis            & 1988-09-19         \\
			\hline
		\end{tabular}
		\caption{Example of table.}
		\label{tab:2}
	\end{table}
\end{frame}

% ************************
% ***** BIBLIOGRAPHY *****
% ************************
\begin{frame}[allowframebreaks]{Bibliography}
	Here is a reference to a source. For example, \textit{Momentum Contrast for Unsupervised Visual Representation Learning by He et al. (2020)} \cite{he2020momentum} is a placeholder text for a bibliography entry.

	\framebreak

	\scriptsize
	\printbibliography
\end{frame}
% ***************
% ***** END *****
% ***************

\end{document}
